\section{Methodology}
This work simulates different transition scenarios to advanced reactors
requiring \gls{HALEU} for fuel, then quantifies and compares the front-end 
material requirements of each scenario. 

Five different fuel cycle scenarios are modeled using \Cyclus \cite{huff_fundamental_2016},
an agent based fuel cycle simulator. The first simulation models only the \glspl{LWR}
currently deployed in the U.S. The next two simulations model no-growth 
transitions to the \gls{USNC} \gls{MMR}\textsuperscript{TM} or the X-Energy 
Xe-100\textsuperscript{TM} reactor. Then the final 
two simulations model a 1\% annual growth transition to either the \gls{USNC} \gls{MMR}
or the X-Energy Xe-100 reactor. Table \ref{tab:simulations} summarizes each
of the simulations.

\begin{table}[ht]
        \centering
        \caption{Summary of the fuel cycle scenarios}
        \label{tab:simulations}
        \begin{tabular}{c c c}
                \hline
                Simulation Number & Reactors Present & Growth \\\hline
                1 & \glspl{LWR} & N/A \\
                2 & \glspl{LWR} and \gls{USNC} \gls{MMR} & None \\
                3 & \glspl{LWR} and X-Energy Xe-100 reactor& None \\
                4 & \glspl{LWR} and \gls{USNC} \gls{MMR}& 1\% \\
                5 & \glspl{LWR} and X-Energy Xe-100 reactor& 1\% \\\hline

        \end{tabular}
\end{table}

We used the \gls{IAEA} \gls{PRIS} data base \cite{noauthor_power_1989} to obtain 
data about the currently deployed \glspl{LWR}. This database provides the 
grid connection date of each reactor deployed, power level, and the decommission 
date for any reactor closed down before December 2020. Any reactor still in 
operation in December 2020 is assumed to operate for 60 years after it's 
grid connection date. Only reactors with a power level above 400 MWe were included 
in the simulations to ensure that no research or experimental reactors are used. 

We used a variety of open source documents to obtain data about the advanced reactors
\cite{harlan_x-energy_2018, hussain_advances_2018, mitchell_usnc_2020}. This includes 
the power output, enrichment level, fuel form, reactor lifetime, and cycle time. 
The Xe-100 reactor has a much higher power output, requires a higher enrichment 
level, and has a longer reactor lifetime than the \gls{MMR}. However, the 
\gls{MMR} has a longer cycle time than the Xe-100 since it does not require 
refueling once it is operational. Both of the advanced reactors require fuel 
comprised of \gls{TRISO} particles, but in different forms. 

\begin{table}[ht]
        \caption{Mico-reactor design specifications}
        \label{tab:reactor_summary}
        \begin{tabular}{l c c c}
            \hline
            Design Criteria & \gls{USNC} \gls{MMR}\textsuperscript{TM} & 
                X-Energy Xe-100\textsuperscript{TM} \\\hline
            Reactor type & Modular HTGR & Modular HTGR \\
            Power Output (MWth) & 15 & 200 \\
            Enrichment (\% $^{235}U$) & 13 & 15.5 \\
            Cycle Length (years) & 20 & online refuel\\
            Fuel form & \gls{TRISO} compacts & \gls{TRISO} pebbles\\
            Reactor Lifetime & 20 years & 60 years \\
            Mass of urnaium per refueling (kg) & 1912.9& 223.87 \\
            \hline
        \end{tabular}
    \end{table}

Each of the simulations run from 1965 to 2090, with the transition beginning 
in 2025. Therefore, in the no growth scenarios the power demand is held constant 
at the power produced in 2025. Fuel cycle facilities included in the simulations
range from the uranium mine to the final disposal of the spent fuel. 