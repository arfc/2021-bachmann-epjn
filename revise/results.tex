\section{Results}
The results presented for each scenario include the energy produced, reactor 
deployment schedule, enriched
uranium mass, \gls{HALEU} mass, and \gls{SWU} capacity required as a function 
of time. 

\subsection{Scenario 1}
The energy supplied by the \glspl{LWR} and the number of \glspl{LWR}
deployed as a function of time are shown in Figure \ref{fig:energy_rx_1}. 
\glspl{LWR} are first deployed in August of 1967. The last 
\gls{LWR} is deployed in June of 2016 and decommissioned in July 2076. The 
maximum number of 
\glspl{LWR} deployed at one time is 109. The energy produced by the 
\glspl{LWR} follows the number of reactors deployed. The maximum energy 
produced by the \glspl{LWR} is 102.46 GWe-y, and they produce 91.82 GWe-y 
in 2025.

\begin{figure}
    \centering 
    \includegraphics[width=\textwidth]{../figures/energy_scenario1.png}
    \caption{Energy supplied by \glspl{LWR} compared to the number of 
    \glspl{LWR} deployed in Scenario 1.}
    \label{fig:energy_rx_1}
\end{figure}

Next, Figure \ref{fig:fuel_1} shows the mass of enriched uranium sent to 
the \glspl{LWR} at each time step. The \glspl{LWR} are 
defined to have an 18 month cycle length, with a third of the uranium 
mass replaced each refueling outage. New reactors 
are deployed with an entire core of uranium, leading 
to increases in the mass of uranium sent to the reactors at a single time 
step, such 
as in 1983 and 2016. An average of 96.2 MTU/month and a maximum of 513.7 MTU 
are sent to the \glspl{LWR}. Prior to 2025, the average mass
of enriched uranium sent to the \glspl{LWR} is 157.6 MTU/month. A total 
of 30,635.0 MTU are sent to \glspl{LWR} after 2025 in this scenario. 

\begin{figure}
    \centering 
    \includegraphics[width=\textwidth]{../figures/fuelsupply_scenarios_1.png}
    \caption{Mass of enriched uranium sent to reactors in Scenario 1.}
    \label{fig:fuel_1}
\end{figure}

Finally, Figure \ref{fig:swu_1} shows the \gls{SWU} capacity to produce 
fuel for the scenario. The \gls{SWU} capacity required as a function of 
time follows the mass of uranium sent to the reactors, because the \gls{SWU}
is calculated based on those transactions. This scenario requires an 
average of 0.74$\times 10^6$ kg-\gls{SWU}/month and a maximum of 
3.95$\times 10^6$ kg-\gls{SWU}. Prior to 2025, this scenario requires 
an average of 1.21$\times 10^6$ kg-\gls{SWU}/month to enrich the uranium 
sent to the \glspl{LWR}.
After 2025, an average of 0.302$\times 10^6$ kg-\gls{SWU} is required to 
enrich the uranium sent to the \glspl{LWR}.
A total of 11.1$\times 10^8$ kg-SWU are required to enrich uranium for the 
reactors in this scenario, and a total of 2.36$\times 10^8$ kg-SWU is required 
to enrich uranium sent to \glspl{LWR} after 2025. 

\begin{figure}
    \centering
    \includegraphics[width=\textwidth]{../figures/totalswu_scenarios_1.png}
    \caption{Total \gls{SWU} capacity required to produce fuel sent to the 
    reactors at each time step in Scenario 1.}
    \label{fig:swu_1}
\end{figure}

The other scenarios in this work apply the deployment and decommissioning 
schedule of the \glspl{LWR} in this scenario, and therefore the material 
requirements of the transition scenarios prior to 2025 are the 
same as those of this scenario.  

\subsection{Scenario 2}
The energy produced by each type of reactor, the transition energy demand, 
and the deployment schedule of the \glspl{MMR} in Scenario 2 are shown in 
Figure \ref{fig:energy_rx_2}. The first \glspl{MMR} are deployed in 
October 2031, and the maximum number deployed at one time is 
9,182. This is about an order of magnitude greater than the \glspl{LWR}
deployed in 2025.

Once the transition begins in 2025, the energy demand of the scenario is 
not met every year. The first of energy deficit is between 2030-2050, with a 
maximum deficit 
of 5.7855 GWe-y in 2032. Other periods in which the energy demand is not met 
correspond to the decommissioning of \glspl{MMR} at the end of their lifetime 
and the deployment of new reactors, such as from 2062-2069. 
The deployment of \glspl{MMR} in 2031 contributes to the inability to 
meet the energy demand of the scenario between 2030-2050, because the energy 
produced by 
\glspl{LWR} decreases to 89.35 GWe-y in 2030, despite a demand of 91.82
GWe-y, before the \Cycamore ManagerInst deploys \glspl{MMR}.
Therefore, the institution deploys the reactors in a reactionary fashion to 
past energy production, as opposed to in response to forecasted energy 
production.

\begin{figure}
    \centering 
    \includegraphics[width=\textwidth]{../figures/energy_scenario2.png}
    \caption{Energy supplied by each type of reactor compared to the number of 
    \glspl{MMR} deployed in Scenario 2.}
    \label{fig:energy_rx_2}
\end{figure}

Next, Figure \ref{fig:fuel_2} shows the mass of enriched uranium sent to all 
reactors in the scenario and just to the \glspl{MMR}. The \glspl{MMR} 
do not require refueling during their lifetime, so uranium is  
sent to these reactors only when they are deployed. This fueling scheme 
causes the increases in uranium sent to \glspl{MMR} and the periods of 
no uranium sent to them. All of the reactors receive an average of 104.94 
MTU/month and a maximum of 781.41 MTU after 2025. 
This average is less than the average mass of enriched uranium 
sent to the \glspl{LWR} in Scenario 1, but the maximum exceeds the maximum 
mass of enriched uranium sent to \glspl{LWR} in Scenario 1 by 267.71 MTU.
The \glspl{MMR} receive an average of 73.12 MTU/month and a maximum of 719.25 
MTU once they are deployed in 2031. The average 
mass of enriched uranium sent to the 
\glspl{MMR} is less than the average mass of uranium sent to the \glspl{LWR}
prior to 2025. This is despite a greater maximum and multiple peaks larger
than the
average mass sent to the \glspl{LWR}. The smaller average is because of the 
lack of refueling and the multiple time steps when additional 
\glspl{MMR} are not deployed. The average mass of uranium required to fuel 
the \glspl{MMR} is more than what is required by \gls{EG} 02 
\cite{wigeland_nuclear_2014}, despite producing less power. 
After 2025, all of the reactors receive a total of 81,747.7 MTU. The  
\glspl{MMR} receive a total of 51,112.7 MTU. These totals show that most 
of the uranium produced after 2025 is for use in the advanced reactors, 
which is consistent with the reactor deployment and decommissioning schedule. 

\begin{figure}
    \centering
    \begin{subfigure}{0.45\textwidth}
        \centering
        \includegraphics[scale=0.4]{../figures/fuelsupply_scenarios_2.png}
        \caption{Mass of enriched uranium sent to all reactors.}
        \label{fig:totalfuel_2}
    \end{subfigure}
    \hspace{0.8cm}
    \begin{subfigure}{0.45\textwidth}
        \centering
        \includegraphics[scale=0.4]{../figures/advancedRX_fuelsupply_scenarios_2.png}
        \caption{Mass of enriched uranium sent to \glspl{MMR}.}
        \label{fig:haleu_2}
    \end{subfigure}
    \caption{Enriched uranium mass sent to reactors in Scenario 2.}
    \label{fig:fuel_2}
\end{figure}

Figure \ref{fig:swu_2} shows the \gls{SWU} capacity needed to 
enrich uranium for all reactors in the scenario, and to enrich uranium for 
just the \glspl{MMR}. The \gls{MMR} fleet requires a greater \gls{SWU} 
capacity than the \gls{LWR} fleet 
prior to 2025 because the \glspl{MMR} require uranium at a higher enrichment 
level. Enriching uranium for the \glspl{MMR} requires an average of 
2.07$\times 10^6$ kg-\gls{SWU}/month and a maximum of 20.3$\times 10^6$ 
kg-\gls{SWU}. These values are both more than the 
average and maximum \gls{SWU} capacity needed to enrich uranium for the 
\glspl{LWR} prior to 2025. Enriching uranium for all reactors after 2025 
requires a total of 16.8$\times 10^8$ kg-SWU and enriching uranium for 
just the \glspl{MMR} requires 14.4$\times 10^8$ kg-SWU. 

\begin{figure}
    \centering
    \begin{subfigure}{0.45\textwidth}
        \centering
        \includegraphics[scale=0.4]{../figures/totalswu_scenarios_2.png}
        \caption{\gls{SWU} required to enrich uranium sent to all reactors at each time step.}
        \label{fig:totalswu_2}
    \end{subfigure}
    \hspace{0.8cm}
    \begin{subfigure}{0.45\textwidth}
        \centering
        \includegraphics[scale=0.4]{../figures/haleuSWU_scenarios_2.png}
        \caption{\gls{SWU} required to enrich uranium sent to \glspl{MMR} at each time step.}
        \label{fig:haleuswu_2}
    \end{subfigure}
    \caption{\gls{SWU} required to enrich natural uranium in Scenario 2.}
    \label{fig:swu_2}
\end{figure}

\subsection{Scenario 3}
For Scenario 3, Figure \ref{fig:energy_rx_3} shows the number of Xe-100 
reactors deployed, the energy produced by each reactor type, and the 
energy demand. Xe-100 reactors are deployed in October 2031, which is the 
same as when \glspl{MMR}
are deployed in Scenario 2. Scenario 3 deploys a maximum of 1,225 Xe-100 
reactors. Nine times more \glspl{MMR} are needed than Xe-100 reactors to 
meet the no-growth energy transition.

Scenario 3 exhibits the same deficit between the energy produced and 
demand between 2030-2050 that was observed in Scenario 2. However, the 
energy produced does not differ from the energy demand by more than 1 GWe-yr 
after 2050 because the Xe-100 reactors have a longer lifetime and the 
simulation does not include the replacement of the Xe-100s. After 2050, the 
maximum difference between the energy produced and demand is 0.057 GWe-y. 

\begin{figure}
    \centering 
    \includegraphics[width=\textwidth]{../figures/energy_scenario3.png}
    \caption{Energy supplied by each type of reactor compared to the number of 
    Xe-100s deployed in Scenario 3.}
    \label{fig:energy_rx_3}
\end{figure}

Comparing the mass of uranium sent to all of the reactors and just the Xe-100 
reactors, shown in Figure \ref{fig:fuel_3}, the Xe-100 reactors 
require less fuel at each time step than what is sent to the \glspl{LWR}, 
despite there being more Xe-100 reactors than \glspl{LWR}. An average of 
74.98 MTU/month and a maximum of 342.58 MTU are sent to all of the reactors 
in this scenario starting in 2025. An average of 39.74 
MTU/month and a maximum of 105.67 MTU are sent to the Xe-100 reactors in 
this scenario. Both metrics are less than the 
average and maximum masses of enriched uranium  
in Scenario 1, and the \gls{HALEU} mass sent to the \glspl{MMR} in 
Scenario 2. The average mass required to fuel the Xe-100 reactors is less 
than the fuel mass required by the \glspl{HTGR} in \gls{EG} 02 
\cite{wigeland_nuclear_2014}. A total of 58,410.1 MTU and 27,775.1 MTU are sent to 
all reactors after 2025 and the advanced reactors in the scenario, respectively, 
showing
that this transition scenario requires less uranium than Scenario 2. 

\begin{figure}
    \centering
    \begin{subfigure}{0.45\textwidth}
        \centering
        \includegraphics[scale=0.4]{../figures/fuelsupply_scenarios_3.png}
        \caption{Mass of enriched uranium sent to all reactors.}
        \label{fig:totalfuel_3}
    \end{subfigure}
    \hspace{0.8cm}
    \begin{subfigure}{0.45\textwidth}
        \centering
        \includegraphics[scale=0.4]{../figures/advancedRX_fuelsupply_scenarios_3.png}
        \caption{Mass of enriched uranium sent to Xe-100s.}
        \label{fig:haleu_3}
    \end{subfigure}
    \caption{Enriched uranium mass sent to reactors in Scenario 3.}
    \label{fig:fuel_3}
\end{figure}

Figure \ref{fig:swu_3} shows the \gls{SWU} capacity needed to enrich 
uranium for all of the reactors in the scenario, and for the uranium sent 
to just the Xe-100 reactors. The average \gls{SWU} capacity needed to enrich uranium 
in Scenario 3 after 2025 is similar to the capacity to enrich uranium prior to 2025, 
despite the increased enrichment level of uranium 
sent to the Xe-100 reactors. This is because the Xe-100 reactors receive 
a smaller average mass of enriched uranium at each time step. Enriching uranium 
for the Xe-100 reactors requires an average of 
1.37$\times 10^6$ kg-\gls{SWU}/month and a maximum of 3.64$\times 10^6$
kg-\gls{SWU}. A total of 11.9$\times 10^8$ kg-SWU and 9.57$\times 10^8$
kg-SWU are required 
to enrich uranium for all reactors after 2025 and the advanced reactors in the 
scenario, respectively. 

\begin{figure}
    \centering
    \begin{subfigure}{0.45\textwidth}
        \centering
        \includegraphics[scale=0.4]{../figures/totalswu_scenarios_3.png}
        \caption{\gls{SWU} required to enrich uranium sent to all reactors.}
        \label{fig:totalswu_3}
    \end{subfigure}
    \hspace{0.8cm}
    \begin{subfigure}{0.45\textwidth}
        \centering
        \includegraphics[scale=0.4]{../figures/haleuSWU_scenarios_3.png}
        \caption{\gls{SWU} required to enrich uranium sent to Xe-100s.}
        \label{fig:haleuswu_3}
    \end{subfigure}
    \caption{\gls{SWU} required to enrich natural uranium in Scenario 3.}
    \label{fig:swu_3}
\end{figure}

\subsection{Scenario 4}
Figure \ref{fig:energy_rx_4} shows the number of \glspl{MMR} deployed, the
energy produced by each type of reactor, and the energy demand of this
scenario with 1\% growth. \glspl{MMR} are deployed in January 2030, and the maximum 
number of \glspl{MMR} is 17,496. 

There is a deficit between the energy produced and  
demand from 2026-2046, with a maximum difference of 4.51 GWe-y in 2032.
There are no other times in which the energy demand is not met by a 
significant amount (more than 1 GWe-y), including when the \glspl{MMR} are 
decommissioned. After 2047 electricity is generally produced in surplus, 
up to 1.64 GWe-y. 

\begin{figure}
    \centering 
    \includegraphics[width=\textwidth]{../figures/energy_scenario4.png}
    \caption{Energy supplied by each type of reactor compared to the number of 
    \glspl{MMR} deployed in Scenario 4.}
    \label{fig:energy_rx_4}
\end{figure}

Figure \ref{fig:fuel_4} shows the mass of enriched uranium sent to all the 
reactors in the scenario, and to just the \glspl{MMR} 
in the scenario. An average of 144.36 MTU/month and a maximum of 796.71 MTU
are sent to all the reactors starting in 2025 in this scenario. An average of 
113.64 MTU/month and a maximum of 782.38 MTU are sent to just the \glspl{MMR}
in the scenario once they are deployed. The mass of \gls{HALEU}
required by the \glspl{MMR} in this scenario is more than  
in Scenario 2 due to the increased energy 
demand and the additional \glspl{MMR} deployed. The 
average mass of \gls{HALEU} sent to the \glspl{MMR} in this scenario is 
slightly less than the average mass sent to the \glspl{LWR}. 
A total of 112,453.6 MTU and 81,818.6 MTU are required for all of the 
reactors after 2025 and the advanced reactors in the scenario, respectively. 

\begin{figure}
    \centering
    \begin{subfigure}{0.45\textwidth}
        \centering
        \includegraphics[scale=0.4]{../figures/fuelsupply_scenarios_4.png}
        \caption{Mass of enriched uranium sent to all reactors.}
        \label{fig:totalfuel_4}
    \end{subfigure}
    \hspace{0.8cm}
    \begin{subfigure}{0.45\textwidth}
        \centering
        \includegraphics[scale=0.4]{../figures/advancedRX_fuelsupply_scenarios_4.png}
        \caption{Mass of enriched uranium sent to \glspl{MMR}.}
        \label{fig:haleu_4}
    \end{subfigure}
    \caption{Enriched uranium mass sent to reactors in Scenario 4.}
    \label{fig:fuel_4}
\end{figure}

Figure \ref{fig:swu_4} shows the \gls{SWU} capacity required to 
enrich uranium for all the reactors in the scenario, and 
only for the \glspl{MMR}. For the same reasons described for 
Scenario 2, the \gls{SWU} capacity required to enrich uranium 
for the \glspl{MMR} is greater than the capacity needed to 
enrich uranium for the \glspl{LWR}. An average of 3.21$\times 10^6$ 
kg-\gls{SWU}/month and a maximum of 22.1$\times 10^6$ kg-\gls{SWU}
are required to enrich the uranium that is sent to \glspl{MMR}. These values 
are larger than the \gls{SWU} 
capacity required to enrich uranium for the \glspl{MMR} in 
Scenario 2. The average \gls{SWU} capacity to enrich uranium for 
the \glspl{MMR} is slightly greater than the average \gls{SWU} 
capacity required to enrich uranium for the \glspl{LWR} before 2025. 
A total of 25.5$\times 10^8$ kg-SWU and 23.1$\times 10^8$
kg-SWU are required to enrich uranium for all reactors after 2025 and the advanced 
reactors in the scenario, respectively. 

\begin{figure}
    \centering
    \begin{subfigure}{0.45\textwidth}
        \centering
        \includegraphics[scale=0.4]{../figures/totalswu_scenarios_4.png}
        \caption{\gls{SWU} required to enrich uranium sent to all reactors.}
        \label{fig:totalswu_4}
    \end{subfigure}
    \hspace{0.8cm}
    \begin{subfigure}{0.45\textwidth}
        \centering
        \includegraphics[scale=0.4]{../figures/haleuSWU_scenarios_4.png}
        \caption{\gls{SWU} required to enrich uranium sent to \glspl{MMR}.}
        \label{fig:haleuswu_4}
    \end{subfigure}
    \caption{\gls{SWU} required to enrich natural uranium in Scenario 4.}
    \label{fig:swu_4}
\end{figure}


\subsection{Scenario 5}
Figure \ref{fig:energy_rx_5} shows the number of Xe-100 reactors, the 
energy produced by each type of reactor, and the energy demand of the 
scenario. The Xe-100 reactors are deployed in January 2030, the 
same time 
\glspl{MMR} are deployed in Scenario 4. The maximum number of Xe-100 
reactors deployed in the scenario is 2,339, which is about 15,000 reactors 
fewer than the \glspl{MMR} required to meet the same energy demand. 

The energy produced is less than the energy demand from 
2026-2046, the same deficit that is observed in Scenario 4. After this 
initial difference, there are no further significant (more than 
1 GWe-y) differences between 
the energy produced and demand. For most years after 2046 there 
is a surplus of energy, up to 1.64 GWe-y. 

\begin{figure}
    \centering 
    \includegraphics[width=\textwidth]{../figures/energy_scenario5.png}
    \caption{Energy supplied by each type of reactor compared to the number of 
    Xe-100s deployed in Scenario 5.}
    \label{fig:energy_rx_5}
\end{figure}

Figure \ref{fig:fuel_5} shows the mass of enriched uranium sent the all  
the reactors in the scenario and the \gls{HALEU} mass sent just to the 
Xe-100 reactors. There is a clear increase in the mass of \gls{HALEU} sent 
to the Xe-100 reactors as time goes on, but this mass is still  
low compared to the mass of enriched uranium sent to the \glspl{LWR} in 
the scenario. An average of 95.46 MTU/month and a maximum of 347.28 MTU 
are sent to all of the reactors in this scenario after 2025. These 
values are both less than the uranium mass sent to the 
\glspl{LWR} prior to 2025. An average of 
60.73 MTU/month and a maximum of 123.80 MTU are sent to the Xe-100 reactors. 
These metrics are all less than what is observed 
for fueling the \glspl{MMR} in Scenario 4. A total of 74,361.76 MTU and 
43,726.74 MTU are required to fuel all reactors after 2025 and the advanced reactors 
in the scenario, respectively.


\begin{figure}
    \centering
    \begin{subfigure}{0.45\textwidth}
        \centering
        \includegraphics[scale=0.4]{../figures/fuelsupply_scenarios_5.png}
        \caption{Mass of enriched uranium sent to all reactors.}
        \label{fig:totalfuel_5}
    \end{subfigure}
    \hspace{0.8cm}
    \begin{subfigure}{0.45\textwidth}
        \centering
        \includegraphics[scale=0.4]{../figures/advancedRX_fuelsupply_scenarios_5.png}
        \caption{Mass of enriched uranium sent to Xe-100s.}
        \label{fig:haleu_5}
    \end{subfigure}
    \caption{Enriched uranium mass sent to reactors in Scenario 5.}
    \label{fig:fuel_5}
\end{figure}

Finally, Figure \ref{fig:swu_5} shows the \gls{SWU} capacity required
to enrich the uranium sent to all of the reactors, and just the Xe-100
reactors. The \gls{SWU} capacity needed 
to enrich uranium for the Xe-100 reactors starts out at a similar 
amount as the \glspl{LWR}, but 
increases as the energy demand and the number of reactors increases. 
The \gls{SWU} capacity required to enrich uranium 
for the Xe-100 reactors becomes greater than the capacity 
required to enrich uranium for \glspl{LWR}, despite the Xe-100 
reactors requiring a lesser mass of uranium. This is because the 
Xe-100 requires fuel at a greater enrichment level. An average of 
2.09$\times 10^6$ kg-\gls{SWU}/month and a maximum of 
4.26$\times 10^6$ kg-\gls{SWU} are required to enrich the uranium sent 
to the Xe-100 reactors. The average \gls{SWU} capacity 
required to enrich uranium for the Xe-100 reactors is lower 
than the average capacity needed to enrich uranium for the \glspl{MMR}
in Scenario 4, but the maximum capacity required for this scenario is much 
less than in Scenario 4. These values are slightly greater 
than what is observed to enrich uranium for the \glspl{LWR}
prior to 2025. A total of 17.4$\times 10^8$ kg-SWU and 15.1$\times 10^8$
kg-SWU are required to enrich uranium for all reactors after 2025 and the advanced 
reactors in the scenario.  

\begin{figure}
    \centering
    \begin{subfigure}{0.45\textwidth}
        \centering
        \includegraphics[scale=0.4]{../figures/totalswu_scenarios_5.png}
        \caption{Total \gls{SWU} required to enrich uranium sent to all reactors at each time step.}
        \label{fig:totalswu_5}
    \end{subfigure}
    \hspace{0.8cm}
    \begin{subfigure}{0.45\textwidth}
        \centering
        \includegraphics[scale=0.4]{../figures/haleuSWU_scenarios_5.png}
        \caption{\gls{SWU} required to enrich uranium sent to Xe-100s at each time step.}
        \label{fig:haleuswu_5}
    \end{subfigure}
    \caption{\gls{SWU} required to enrich natural uranium in Scenario 5.}
    \label{fig:swu_5}
\end{figure}

\subsection{Scenario comparisons}
The mass of uranium sent to all reactors for each scenario and to just the 
advanced reactors is shown in Figure \ref{fig:fuel_all}. Fueling the Xe-100 
reactors in Scenarios 3 and 5 requires less uranium than fueling the \glspl{LWR}
prior to 2025
at each time step. Fueling the \glspl{MMR} in Scenarios 2 and 4 requires the 
most uranium at any single time step, but the average uranium mass is less 
than what is required to fuel the \glspl{LWR} prior to 2025. This is because 
of multiple time steps in which no or a small mass (less than 20 MTU) 
of uranium is sent to \glspl{MMR}, which offset the timesteps that require 
a large mass of uranium (more than 200 MTU) are sent to the \glspl{MMR}. 

\begin{figure}
    \centering
    \begin{subfigure}{0.45\textwidth}
        \centering
        \includegraphics[scale=0.4]{../figures/fuelsupply_scenarios_all.png}
        \caption{Total uranium mass sent to all reactors at each time step.}
        \label{fig:totalfuel_all}
    \end{subfigure}
    \hspace{0.8cm}
    \begin{subfigure}{0.45\textwidth}
        \centering
        \includegraphics[scale=0.4]{../figures/advancedRX_fuelsupply_scenarios_2-5.png}
        \caption{Uranium mass sent to advanced reactors at each time step.}
        \label{fig:haleufuel_all}
    \end{subfigure}
    \caption{Uranium mass supplied to reactors in all scenarios.}
    \label{fig:fuel_all}
\end{figure}

Comparing the cumulative total mass of enriched uranium required in each scenario, 
Figure \ref{fig:cumulativeU_all}, deploying the \gls{MMR} 
requires more uranium than deploying the Xe-100 reactor for the same 
transition scenario. Deploying the \gls{MMR} in a no growth transition 
scenario requires more uranium than deploying Xe-100 reactors in a 1\% 
growth transition. 

\begin{figure}
    \centering 
    \includegraphics[scale=0.4]{../figures/fuelsupplytotal_scenarios_all.png}
    \caption{Cumulative total uranium mass sent to all reactors in each scenario.}
    \label{fig:cumulativeU_all}
\end{figure}

Figure \ref{fig:swu_all} compares the \gls{SWU} capacity required to enrich uranium for 
all reactors and the advanced reactors in each scenario. The \gls{SWU} capacity required 
to enrich uranium in Scenarios 1, 3, and 5 are similar in magnitude. Scenarios 2 and 4
appear to require a greater \gls{SWU} capacity because the calculated \gls{SWU} capacity 
is based on the mass of uranium sent to the reactors at each time step. Therefore, 
the values presented do not reflect the required \gls{SWU} capacity of 
an enrichment facility. 

\begin{figure}
    \centering
    \begin{subfigure}{0.45\textwidth}
        \centering
        \includegraphics[scale=0.4]{../figures/totalswu_scenarios_all.png}
        \caption{Total \gls{SWU} required to enrich uranium sent to all reactors at each time step.}
        \label{fig:totalswu_all}
    \end{subfigure}
    \hspace{1.8cm}
    \begin{subfigure}{0.45\textwidth}
        \centering
        \includegraphics[scale=0.4]{../figures/haleuSWU_scenarios_all.png}
        \caption{\gls{SWU} required to enrich uranium sent to advanced reactors at each time step.}
        \label{fig:haleuswu_al}
    \end{subfigure}
    \caption{\gls{SWU} required to enrich natural uranium in all scenarios.}
    \label{fig:swu_all}
\end{figure}

An example of a possible facility for enriching uranium for the \glspl{MMR} in 
Scenario 2 would 
have a throughput of 66 MTU/month, and a \gls{SWU} capacity of 2.28$\times 10^6$
kg-SWU/month. A facility of this size and capacity would be able to meet the 
enriched uranium demand, as Figure \ref{fig:enrich_storage}
shows. In this scenario, the enrichment facility goes online in 2025 and operates 
through 2090. The stored enriched uranium mass at the facility never goes below 0, 
but this example does not 
account for any time required to fabricate the fuel before shipment to the reactor.
This theoretical example shows how a constant-capacity \gls{HALEU} enrichment 
facility can meet the enriched uranium requirements of the scenario, but further 
optimization of such a facility or the exploration of how other methods can be
used meet the enriched uranium demand is left to future work. 

\begin{figure}
    \centering
    \includegraphics[width=\textwidth]{../figures/potential_uranium_stockpile.png}
    \caption{Net enriched uranium mass stored at a theoretical enrichment facility with 
    a mass throughput of 66 MTU/month. This model assumes the facility goes online in 2025
    and does not account for any time required for fuel fabrication.}
    \label{fig:enrich_storage}
\end{figure}
