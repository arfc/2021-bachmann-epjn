\section{Results}
The results presented for each scenario include the energy produced, reactor 
deployment schedule, natural
uranium mass, \gls{HALEU} mass, and \gls{SWU} capacity as a function of time. 

\subsection{Scenario 1}
The amount of energy supplied by the \glspl{LWR} and the number of \glspl{LWR}
deployed as a function of time is shown in Figure \ref{fig:energy_rx_1}. 
\glspl{LWR} are first deployed in August of 1967, and the last 
\gls{LWR} is decommissioned in June of 2016. The maximum number of 
\glspl{LWR} deployed at one time is 109. The enery produced by the 
\glspl{LWR} follows with the number of reactors deployed. The maximum energy 
produced by the \glspl{LWR} is 102.46 GWe-y, and they produce 91.82 GWe-y 
in 2025.

\begin{figure}
    \centering 
    \includegraphics[scale=0.5]{figures/energy_scenario1.png}
    \caption{Energy supplied by \glspl{LWR} compared to the number of 
    \glspl{LWR} deployed in Scenario 1.}
    \label{fig:energy_rx_1}
\end{figure}

The next result is the mass of uranium sent to the \glspl{LWR} to fuel 
them, as shown in Figure \ref{fig:fuel_1}. Each of the \glspl{LWR} are 
defined to have an 18 month cycle length, with a third of the uranium 
mass replaced at each refueling outage. However, each when a new reactor 
is deployed, it is deployed with an entire core of uranium which leads 
to the large increases in the mass of urnaium sent to the reactors, such 
as what is observed in 1983 and 2016. A maximum of 513.7 MT of uranium 
is required at one time for this scenario.

\begin{figure}
    \centering 
    \includegraphics[scale=0.5]{figures/fuelsupply_scenarios_1.png}
    \caption{Mass of uranium sent to reactors in Scenario 1.}
    \label{fig:fuel_1}
\end{figure}

Finally, the \gls{SWU} capacity to produce fuel for the scenario, shown in 
Figure \ref{fig:swu_1}. The \gls{SWU} capacity required as a function of 
time follows the mass of uranium sent to the reactors, since the \gls{SWU}
is calculated based on those transactions. A maximum \gls{SWU} capacity of 
3950737.5 kg-\gls{SWU} and an average capacity of 740001.8 kg-\gls{SWU} is 
required for this scenario.

\begin{figure}
    \centering
    \includegraphics[scale=0.5]{figures/totalswu_scenarios_1.png}
    \caption{Total \gls{SWU} capacity required to produce fuel sent to the 
    reactors at each time step in Scenario 1.}
    \label{fig:swu_1}
\end{figure}


Each of the metrics from this scenario, the \gls{LWR} deployment schedule, 
energy supplied by them, uranium mass used to fuel the \glspl{LWR}, and the 
\gls{SWU} capacity needed to produce their fuel are the same for each of 
the other scenarios. 

\subsection{Scenario 2}
The energy produced by each type of reactor, the transition energy demand, 
and the deployment schedule of the \glspl{MMR} for Scenario 2 are shown in 
Figure \ref{fig:energy_rx_2}. Once the transition begins in 2025, there are 
clearly some time periods in which the energy demand of the scenario is 
not met. The first of these is between 2030-2050, with a maximum defecit 
of 5.7855 GWe-y in 2032. Other periods in which the energy demand is not met 
correspond to the decommissioning of \glspl{MMR} at the end of their lifetime 
and the deployment of new reactors, such as from 2062-2069. 

The first \glspl{MMR} are deployed starting in October 2031, and the 
maximum number of \glspl{MMR} deployed at one time in this scenario is 
9,182. The deployment of \glspl{MMR} in 2031 contributes to the inability to 
meet the energy demand of the scenario because the energy produced by 
\glspl{LWR} decreases to 89.35 GWe-y in 2030, before the \glspl{MMR} are 
deployed. Therefore, the reactors are deployed in a reactionary fashion to 
past energy production, as opposed to pre-emptively to forcasted energy 
production. 

\begin{figure}
    \centering 
    \includegraphics[scale=0.5]{figures/energy_scenario2.png}
    \caption{Energy supplied by each type of reactor compared to the number of 
    \glspl{MMR} deployed in Scenario 2.}
    \label{fig:energy_rx_2}
\end{figure}



\begin{figure}
    \centering
    \begin{subfigure}{0.4\textwidth}
        \centering
        \includegraphics[scale=0.3]{figures/fuelsupply_scenarios_2.png}
        \caption{Total mass of uranium sent to all reactors at each time step.}
        \label{fig:totalfuel_2}
    \end{subfigure}
    \begin{subfigure}{0.4\textwidth}
        \centering
        \includegraphics[scale=0.3]{figures/advancedRX_fuelsupply_scenarios_2.png}
        \caption{Total mass of uranium sent to \glspl{MMR} at each time step.}
        \label{fig:haleu_2}
    \end{subfigure}
    \caption{Uranium mass sent to reactors in Scenario 2.}
    \label{fig:fuel_2}
\end{figure}


\begin{figure}
    \centering
    \begin{subfigure}{0.4\textwidth}
        \centering
        \includegraphics[scale=0.3]{figures/totalswu_scenarios_2.png}
        \caption{Total \gls{SWU} required to enrich uranium sent to all reactors at each time step.}
        \label{fig:totalswu_2}
    \end{subfigure}
    \begin{subfigure}{0.4\textwidth}
        \centering
        \includegraphics[scale=0.3]{figures/advancedRX_fuelsupply_scenarios_2.png}
        \caption{\gls{SWU} required to enrich uranium sent to \glspl{MMR} at each time step.}
        \label{fig:haleuswu_2}
    \end{subfigure}
    \caption{\gls{SWU} required to enrich natural uranium in Scenario 2.}
    \label{fig:swu_2}
\end{figure}

\subsection{Scenario 3}
Figure \ref{fig:energy_rx_3} shows the number of Xe-100 reactors deployed, 
the energy produced by each reactor type, and the energy demand of Scenario 3. 
There is the same gap between the energy produced and energy demand between 
2030-2050 that was observed in Scenario 2. However, there are no significant 
differences between the energy produced and energy demand of the scenario 
after 2050 because the Xe-100 reactors have a longer lifetime and there are 
no decreases in energy due to decommissioning of reactors. After 2050, the 
maximum difference between the energy produced and the demand is 0.057 GWe-y. 

Xe-100 reactors are deployed starting in October 2031, same as when \glspl{MMR}
are first deployed in Scenario 2. Combined with the consistency of the energy 
produced in Scenarios 2 and 3, this suggests that the \Cycamore ManagerInst 
deploys reactors on a reactionary basis instead of forecasting, and that there 
is a small buffer  
GrowthRegion  

maximum of 1225 Xe-100 reactors are deployed

\begin{figure}
    \centering 
    \includegraphics[scale=0.5]{figures/energy_scenario3.png}
    \caption{Energy supplied by each type of reactor compared to the number of 
    Xe-100s deployed in Scenario 3.}
    \label{fig:energy_rx_3}
\end{figure}

\begin{figure}
    \centering
    \begin{subfigure}{0.4\textwidth}
        \centering
        \includegraphics[scale=0.3]{figures/fuelsupply_scenarios_3.png}
        \caption{Total mass of uranium sent to all reactors at each time step.}
        \label{fig:totalfuel_3}
    \end{subfigure}
    \begin{subfigure}{0.4\textwidth}
        \centering
        \includegraphics[scale=0.3]{figures/advancedRX_fuelsupply_scenarios_3.png}
        \caption{Total mass of uranium sent to Xe-100s at each time step.}
        \label{fig:haleu_3}
    \end{subfigure}
    \caption{Uranium mass sent to reactors in Scenario 3.}
    \label{fig:fuel_3}
\end{figure}


\begin{figure}
    \centering
    \begin{subfigure}{0.4\textwidth}
        \centering
        \includegraphics[scale=0.3]{figures/totalswu_scenarios_3.png}
        \caption{Total \gls{SWU} required to enrich uranium sent to all reactors at each time step.}
        \label{fig:totalswu_3}
    \end{subfigure}
    \begin{subfigure}{0.4\textwidth}
        \centering
        \includegraphics[scale=0.3]{figures/advancedRX_fuelsupply_scenarios_3.png}
        \caption{\gls{SWU} required to enrich uranium sent to Xe-100s at each time step.}
        \label{fig:haleuswu_3}
    \end{subfigure}
    \caption{\gls{SWU} required to enrich natural uranium in Scenario 3.}
    \label{fig:swu_3}
\end{figure}

\subsection{Scenario 4}

\begin{figure}
    \centering 
    \includegraphics[scale=0.5]{figures/energy_scenario4.png}
    \caption{Energy supplied by each type of reactor compared to the number of 
    \glspl{MMR} deployed in Scenario 4.}
    \label{fig:energy_rx_4}
\end{figure}


\begin{figure}
    \centering
    \begin{subfigure}{0.4\textwidth}
        \centering
        \includegraphics[scale=0.3]{figures/fuelsupply_scenarios_4.png}
        \caption{Total mass of uranium sent to all reactors at each time step.}
        \label{fig:totalfuel_4}
    \end{subfigure}
    \begin{subfigure}{0.4\textwidth}
        \centering
        \includegraphics[scale=0.3]{figures/advancedRX_fuelsupply_scenarios_4.png}
        \caption{Total mass of uranium sent to \glspl{MMR} at each time step.}
        \label{fig:haleu_4}
    \end{subfigure}
    \caption{Uranium mass sent to reactors in Scenario 4.}
    \label{fig:fuel_4}
\end{figure}


\begin{figure}
    \centering
    \begin{subfigure}{0.4\textwidth}
        \centering
        \includegraphics[scale=0.3]{figures/totalswu_scenarios_4.png}
        \caption{Total \gls{SWU} required to enrich uranium sent to all reactors at each time step.}
        \label{fig:totalswu_4}
    \end{subfigure}
    \begin{subfigure}{0.4\textwidth}
        \centering
        \includegraphics[scale=0.3]{figures/advancedRX_fuelsupply_scenarios_4.png}
        \caption{\gls{SWU} required to enrich uranium sent to \glspl{MMR} at each time step.}
        \label{fig:haleuswu_4}
    \end{subfigure}
    \caption{\gls{SWU} required to enrich natural uranium in Scenario 4.}
    \label{fig:swu_4}
\end{figure}
\subsection{Scenario 5}

\begin{figure}
    \centering 
    \includegraphics[scale=0.5]{figures/energy_scenario5.png}
    \caption{Energy supplied by each type of reactor compared to the number of 
    Xe-100s deployed in Scenario 5.}
    \label{fig:energy_rx_5}
\end{figure}

\begin{figure}
    \centering
    \begin{subfigure}{0.4\textwidth}
        \centering
        \includegraphics[scale=0.3]{figures/fuelsupply_scenarios_5.png}
        \caption{Total mass of uranium sent to all reactors at each time step.}
        \label{fig:totalfuel_5}
    \end{subfigure}
    \begin{subfigure}{0.4\textwidth}
        \centering
        \includegraphics[scale=0.3]{figures/advancedRX_fuelsupply_scenarios_5.png}
        \caption{Total mass of uranium sent to Xe-100s at each time step.}
        \label{fig:haleu_5}
    \end{subfigure}
    \caption{Uranium mass sent to reactors in Scenario 5.}
    \label{fig:fuel_5}
\end{figure}


\begin{figure}
    \centering
    \begin{subfigure}{0.4\textwidth}
        \centering
        \includegraphics[scale=0.3]{figures/totalswu_scenarios_5.png}
        \caption{Total \gls{SWU} required to enrich uranium sent to all reactors at each time step.}
        \label{fig:totalswu_5}
    \end{subfigure}
    \begin{subfigure}{0.4\textwidth}
        \centering
        \includegraphics[scale=0.3]{figures/advancedRX_fuelsupply_scenarios_5.png}
        \caption{\gls{SWU} required to enrich uranium sent to Xe-100s at each time step.}
        \label{fig:haleuswu_5}
    \end{subfigure}
    \caption{\gls{SWU} required to enrich natural uranium in Scenario 5.}
    \label{fig:swu_5}
\end{figure}
