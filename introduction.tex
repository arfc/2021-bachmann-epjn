\section{Introduction}

Multiple new reactor designs have been developed in the last few years, many 
of which will require \gls{HALEU} to fuel them. \gls{HALEU} fuel will help 
the reactors achieve higher burnups and longer cycle times than current 
\glspl{LWR}. However, changes to the enrichment level of the fuel is expected 
to change the material requirements of such a fuel cycle, and may limit the 
ability to deploy large numbers of new reactor designs. In previous analysis 
of the fuel cycle impacts of increasing the enrichment of \gls{PWR} fuel above
5\% $^{235}$U shows an increase in material requirements in the front end of 
the fuel cycle such as natural uranium and \gls{SWU} \cite{burns_reactor_2020}.

This work aims to quantify the material resources of the front end of the 
transition to different types of \gls{HALEU}-fueled advanced reactors. 
Materials of interest include the natural uranium, \gls{SWU} capacity, and 
fuel mass as a function of time. The material requirements of each transition 
scenario will be compared to identify any advantages or disadvantages. 